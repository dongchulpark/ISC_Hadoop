
\section{Conclusion}\label{sec:conclusion}

Unlike the traditional CPU-centric computing model, SSD-based In-Storage Computing (ISC) is a new computing paradigm that enables SSDs to play a major role in computation, not just in yet-another-faster HDDs. This SSD-based ISC system offloads main features from a host to an SSD to make the best use of SSD potential capabilities such as its high interval bandwidth and computational power with low power CPUs. 

In this paper, we apply the ISC model to the Hadoop MapReduce framework--a de facto standard distributed computing framework for big data processing. After investigation, we offload Mapper from a host Hodoop system to SSDs and integrate the existing Hadoop MapReduce system with our ISC devices. This system co-design gives rise to the following challenging technical issues: (1) discrepancy in data representation and (2)system interfaces between a host system and a ISC device, (3) Hadoop data split, and (4) feature offloading. To address the data representation and system interface issues, we add a software layer between the host Hadoop framework and our ISC devices. This software interface layer is in charge of converting a file name to LBA information and communicating Java codes with C/C++ codes. The data split issue in HDFS input split data of 64MB (default) is another critical issue. To resolve this, we consult a host Hadoop system to receive the split information and store the split part as a separate data file. Then our ISC Hadoop system process the separate file as new input data. Which part should be offloaded to the ISC device is always a big issue for ISC research and implementation because moving all features to SSDs does not necessarily result in a performance improvement.
A fully distributed mode on a single Hadoop machine is yet-another interesting issue for Hadoop MapReduce research work. Although Hadoop does not support this mode, it can provide a very useful and efficient way by simulating the fully distributed Hadoop clusters only in a single machine. All of our initial studies and analyses have been made on this mode and it can be applied to the real multi-node Hadoop clusters.

Our extensive experiments and results demonstrate that our ISC Hadoop system exhibits a siginficant performance improvement (2$\times$ faster) as well as marvelous energy saving (more than 9$\times$ less) than a typical Hadoop system. Moreover, our deep analyses are provided to delve into our performance gains.

As we mentioned, we classify wordcount-like Hadoop applications as ISC-favorable applications and choose Hadoop wordcount for our experiments. We expect all such applications will show very similar results and remain its verification via a system implementation as our future work. In addition, we can also investigate ISC-nonfavorable Hadoop applications in the future.

