
\begin{abstract}
Solid State Drives (SSDs) have been developed for the replacement of conventional magnetic Hard Disk Drives (HDDs) as a faster storage device. Their high computational capabilities, however, enable SSDs to become a computing node, not just a storage device, which is called In-Storage Computing (ISC). Hadoop MapReduce framework nowadays became a de facto standard for big data processing. 
This paper explores the challenges and opportunities of In-Storage Computing for Hadoop MapReduce framework. 
For this, we co-design Hadoop MapReduce system and ISC device by implementing the Mapper function inside real SSD firmware and offloading it from MapReduce framework in the host system to the SSDs. Then, we make extensive experiments in real Hadoop clusters. 
Our experiment results demonstrate that our ISC Hadoop MapReduce system achieves a remarkable performance gain (2--3$\times$ faster) as well as significant energy saving (8--10$\times$ lower). 

%In this paper, we co-design Hadoop MapReduce system and ISC device in order to apply ISC to the Hadoop MapReduce framework. For this, we move Mapper to SSDs by implementing the Mapper function inside real SSD firmware and offloading it from MapReduce framework in the host system to the SSDs. Then, we make extensive experiments in real Hadoop clusters. 



\end{abstract}

\category{C.0}{Computer Systems Organization}{System architectures}
\terms{Design, Performance, Experimentation}
\keywords{In-Storage Computing, In-Storage Processing, In-Situ Processing, Smart SSD, Hadoop, MapReduce, Big Data}
